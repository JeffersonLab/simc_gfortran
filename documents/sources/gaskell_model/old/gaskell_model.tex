\documentstyle {article}
\begin{document}
\textheight8.5in
\hsize6.0in
\oddsidemargin0.25in
\evensidemargin0in
\topmargin0in
\title{Pion Production in SIMC}
\author{Dave Gaskell}
\date{\today}
\maketitle

\section{Introduction}
SIMC has been shown to be a powerful tool in analyzing elastic scattering,
(e,e'p), and kaon production data.  This report will summarize how we extend
SIMC to handle pion production and point out some of the ambiguities one must 
face in extending models of pion production from hydrogen to nuclei with A$>$1.

\section{Event Generation}
In SIMC, the beam energy is fixed (modulo some small smearing) and the 
scattered electron quantities ($E_e$,$cos{\theta_e}$,$\phi_e$) are thrown flat 
in the lab frame.  For pion production, the pion angles are also thrown flat.
In the case of production from the free proton, the magnitude of the pion 
momentum is fixed by energy and momentum conservation.  In the case of 
production from the Deuteron and Helium-3 the following procedure is followed.
It is assumed that the pion is produced from a single nucleon in the nucleus 
and the other nucleon(s) are non-participating
spectators.  A momentum, $\vec{p}_{fermi}$, is chosen for the target nucleon.  
The magnitude of the nucleon momentum is drawn from a normalized distribution
derived from an appropriate momentum space wave function, while the direction
is thrown flat in $\phi$ and $cos{\theta}$.  The final element that must be
fixed is the target nucleon energy.  In general, the nucleons in a nucleus are
not on mass shell, so the nucleon momentum does not fix its energy.  In SIMC, 
we choose the target nucleon energy to be defined as
$$
E_{target} = M_{D} - \sqrt{m_{spectator}^2+p^2_{fermi}} 
$$
for Deuterium, and
$$
E_{target} = M_{He} - \sqrt{m_{spectator}^2+p^2_{fermi}} - m_{proton}
$$
for Helium.  
For the Deuterium case, this just puts the spectator on mass shell.  In 
Helium-3, we are essentially saying that the nucleus is a Deuteron with a 
stationary proton. Now with the target nucleon energy and momentum fixed, one
can solve for the magnitude of the pion momentum using energy and momentum
conservation.  More specifically,
$$
\omega + E_{target} = E_{\pi} + E_{recoil}
$$
$$
\vec{q} + \vec{p}_{fermi} = \vec{p}_{\pi} + \vec{p}_{recoil}.
$$
Note that it is assumed that the recoiling struck nucleon is now on the mass 
shell (i.e. $E^2_{recoil}=m^2_{recoil}+p^2_{recoil}$).

\section{Physics Model}
\subsection{Proton Target}
In SIMC, each event is weighted by some model cross section calculated
from the generated (vertex) kinematics.  In the models used in SIMC, 
the electroproduction cross section is written as,
$$
\frac{d\sigma}{d\Omega_{e}dE_ed\Omega_{\pi}^{*}} = \Gamma_{v}
\frac{d\sigma}{d\Omega_{\pi}^{*}}
$$
where $\Gamma_v$ is the virtual photon flux defined as
$$
\Gamma_v = \frac{\alpha}{2\pi^2}\frac{E'}{E}\frac{K_{eq}}{Q^2}
\frac{1}{1-\epsilon}
$$
and $\frac{d\sigma}{d\Omega^{*}}$ is the center of mass virtual photon cross 
section.  $K_{eq}$ is the equivalent photon energy (the energy required for a
real photon to excite a nucleon to a mass W) and is defined (Hand 
convention)
$$
K_{eq} = \frac{W^2-m_{n}^2}{2m_{n}},
$$
where $m_n$ is the mass of the nucleon.

The center of mass virtual photon cross section is written in terms of 
transverse, longitudinal, and interference terms,
$$
\frac{d\sigma}{d\Omega_{\pi}^{*}} = \frac{d\sigma_T}{d\Omega_{\pi}^{*}}+
\epsilon\frac{d\sigma_L}{d\Omega_{\pi}^{*}} + \sqrt{2\epsilon(\epsilon+1)}
 \frac{d\sigma_{TL}}{d\Omega_{\pi}^{*}}\cos{\phi_{\pi}^*} +
 \epsilon\frac{d\sigma_{TT}}{d\Omega_{\pi}^{*}}\cos{2\phi_{\pi}^{*}}.
$$
The center of mass cross section is calculated differently depending on 
the model.  For the high W data (W=1.6 GeV), we use Henk Blok's 
parameterization of $d\sigma/dtd\phi$ from Brauel et. al. \cite{Brauel}, as 
well as the so-called MAID model.  MAID is 
somewhat of a black box since we do not have the original code, although the 
paper describing the physics on which it is based is available \cite{Tiator}.
In essence, the MAID model calculates resonant and non-resonant multipoles
constrained by existing data.  The MAID model is also used for the low W
data (W=1.16 GeV) in addition to an older multipole model (which I will call 
Multipole).  While MAID takes $W$,$Q^2$,and $\cos{\theta_{\pi}^*}$ as inputs 
and uses some analytical parameterization to calculate the multipoles, 
Multipole uses a simple look-up table (binned in $K_{eq}$ and $Q^2$).
In MAID and Multipole one ends up with response functions ($R_T$,$R_L$,etc.) 
that must be multiplied by a kinematic factor to get the center of mass cross 
sections.  Modulo factors of two, the kinematic factors are identical for both
models:
$$
\frac{d\sigma_T}{d\Omega^*} = \frac{p_{\pi}W}{2K_{eq}m_n}R_T
$$
$$
\frac{d\sigma_L}{d\Omega^*} = \frac{p_{\pi}W}{K_{eq}m_n}R_L
$$
$$
\frac{d\sigma_{TL}}{d\Omega^*} = 2\frac{p_{\pi}W}{K_{eq}m_n}REAL(R_{TL})
$$
$$
\frac{d\sigma_{TT}}{d\Omega^*} = \frac{p_{\pi}W}{2K_{eq}m_n}R_{TT}.
$$

For pion production from the proton, the above prescription is unambiguous.
Given an event, one simply needs to calculate $\Gamma_v$ in the lab, boost
the pion to the center of mass and calculate $d\sigma/d\Omega_{\pi}^*$.  Then
one must transform the virtual photon cross section back to the lab via a 
Jacobian (see Appendix A).

\subsection{A$>$1}
When A is greater than one, we take the approach that the model cross section
is just the proton cross section with the caveat that the proton (or neutron)
is now allowed to move.  Implementing this alone is difficult enough, but we 
must also face the ambiguities inherent in extending the model to a particle 
that is not on the mass shell.

First we must deal with the virtual gamma flux ($\Gamma_v$). In calculating
any cross section one generally starts with the inverse of the relative 
electron-target flux, v.  Typically
(i.e. for a stationary target) this is just $E_e/p_e$.  When the target is
allowed to move, however, we must account for the relative velocity of the 
target particle.  Thus we must correct the incident flux, v, by a factor
$[1 - \vec{p}_{fermi}\cdot\vec{p}_e/(E_{target}p_e)]$. 
Furthermore, if one takes $K_{eq}$ to be the real photon energy needed to 
excite a mass W from the target nucleon we then might naively calculate,
$$
K_{eq} = \frac{W^2 - (E_{target}^2-p_{fermi}^2)}{2(E_{target}-
\vec{p}_{fermi}  \cdot \vec{q}/|\vec{q}|)}.
$$
$K_{eq}$ in the virtual photon flux cancels when
combined with the center of mass photon cross section.  However, since it is
used in the Multipole model look-up table, the choice of convention is still
important.  It turns out that when the above form of $K_{eq}$ is used in the 
Multipole model, the center of mass cross section does not have the correct 
dependence on W.  For that reason, the same form for $K_{eq}$ as was used
in the stationary nucleon case is used in the moving nucleon case (this gives
the correct W dependence).  Note, however, that W is calculated differently:
$$
W^2 = (E_{target}+\omega)^2 - (\vec{p}_{fermi}+\vec{q})^2.
$$

A more subtle issue is how to treat the ``off-shellness'' of the target 
nucleon.  Recall that in 
the generation of an event, the spectator nucleon is on the mass shell while
the target is not.  However, the models we use generally assume the target is
on mass shell.  We are free to redefine the target nucleon energy such that
the particle is on shell, but then this energy will be inconsistent with
the energy calculated for the pion.  More specifically, when the pion is 
boosted to the center of mass, it will not actually be boosted to a frame 
where the sum of all momenta is zero.  Currently, we choose to treat the 
struck nucleon the same in both event generation and in the physics
models.

Once we choose a convention for the target nucleon energy, it is not always 
possible to implement it in a consistent manner.  Both MAID and Multipole
require that the fundamental amplitudes (squared) be multiplied by various
kinematic factors.  Since these are calculated outside the model, we are free
to choose whatever convention we like.  The amplitudes themselves, however,
are another story. 
For the Multipole model, once one has chosen a prescription
for calculating the ``off-shell'' $K_{eq}$, there are no ambiguities.  The 
MAID model is a bit more complicated, though.  Recall that the MAID model 
takes $W$,$Q^2$,and $cos{\theta^*}$ as its inputs.  It is unknown how it 
determines the fundamental amplitudes, but from the paper describing this 
model it almost certainly requires the center of mass virtual photon energy 
and momentum.  The 
problem arises from the fact that we have calculated $W$ and $cos{\theta^*}$
assuming some off-shell prescription.  When MAID determines $\omega^*$ 
from the input quantities, it assumes the target nucleon is on shell so that 
the virtual photon energy (and momentum), and therefore the
amplitudes it returns may not be consistent with the kinematics that we 
generated for this event.  This is not a problem if we change the event 
generation scheme and put the target nucleon on shell and put the spectator 
off shell, but then at some kinematic settings, the agreement between data and
SIMC suffers in regard to distribution offsets.  In addition, the spectator 
nucleon is now off shell in the final state.

\section{Appendix A - The Jacobian}
In calculating the cross section section for a given event in SIMC, we 
generally use models that describe the virtual photon cross section in the
center of mass ($d\sigma/d\Omega^*$) multiplied by a virtual photon flux factor
($\Gamma_v$).  Thus, we must transform the virtual photon cross section to the
lab in order to properly weight the event.

The first step is to transform $d\sigma/d\Omega^*$ to $d\sigma/dtd\phi^*$.  
This transformation actually reduces to complexity of the Jacobian somewhat.
The invariant t is written,
$$
t = -Q^2 + m_{\pi}^2 - 2\omega E_{\pi} + 2p_{\pi}qcos{\theta}.
$$
In the center of mass, all energy and momenta do not depend on $\theta$, so
$$
\frac{dt}{d\cos{\theta^*}} = 2p_{\pi}^*q^*
$$
and
$$
\frac{d\sigma}{dtd\phi^*} = \frac{d\sigma}{d\Omega^*}\frac{d\cos{\theta^*}}{dt}
= \frac{1}{2p_{\pi}^*q^*} \frac{d\sigma}{d\Omega^*}.
$$

Next, we need to transform from $(t,\phi^*)$ to $(\cos{\theta}\phi)$.  For that
we need to calculate a Jacobian,
$$J=
\left|\matrix{
\frac{\partial{t}}{\partial{\cos{\theta}}}
&\frac{\partial{t}}{\partial{\phi}} \cr
\frac{\partial{\phi^*}}{\partial{\cos{\theta}}} & 
\frac{\partial{\phi^*}}{\partial{\phi}} \cr
}\right|
$$
First, I will note a couple of useful derivatives.  These are big and ugly - 
the derivation of which are left to the reader as an exercise.  
One derivative that will appear repeatedly is $\partial{p_{\pi}}/
\partial{\cos{\theta_{q\pi}}}$.
$$
\frac{\partial{p_{\pi}}}{\partial{\cos{\theta_{q\pi}}}} = 
\frac{ p_{\pi}\sqrt{q+p_N\cos{\theta_{qN}}-p_{\pi}\cos{\theta_{q\pi}}} + 
p_{\pi}^2\cos{\theta_{q\pi}} - 
p_N\sin{\theta_{qN}}p_{\pi}\frac{\cos{\theta_{q\pi}}}{\sin{\theta_{q\pi}}}
F(\phi_{qN},\phi_{q\pi})}
{p_{\pi}\sin^2{\theta_{q\pi}}-p_N\sin{\theta_{qN}}\sin{\theta_{q\pi}}
F(\phi_{qN},\phi_{q\pi}) + 
(\omega+E_N-E_{\pi})\frac{p_{\pi}}{E_{\pi}} - 
\cos{\theta{q\pi}}\sqrt{q+p_N\cos{\theta_{qN}}-p_{\pi}\cos{\theta_{q\pi}}}}
$$
where $F(\phi_{qN},\phi_{q\pi}) =  
\cos{\phi_{q\pi}}\cos{\phi_{qN}}+\sin{\phi_{q\pi}}\sin{\phi_{qN}}$.  Another 
useful derivative is $\partial{p_{\pi}}/\partial{\phi_{q\pi}}$.
$$
\frac{\partial{p_{\pi}}}{\partial{\phi_{q\pi}}} = 
\frac{p_N\sin{\theta_{qN}}p_{\pi}\sin{\theta_{q\pi}}G(\phi_{qN},\phi_{q\pi})/
(q+p_N\cos{\theta_{qN}}-p_{q\pi}\cos{\theta_{q\pi}})}{\cos{\theta_{q\pi}} +
\frac{1}{\sqrt{q+p_N\cos{\theta_{qN}}-p_{\pi}\cos{\theta_{q\pi}}}}
(p_N\sin{\theta_{qN}}\sin{\theta_{q\pi}}F(\phi_{qN},\phi_{q\pi}) -
p_{\pi}\sin^2{\theta_{q\pi}} - (\omega+E_N-E_{\pi})\frac{p_{\pi}}{E_{\pi}})}
$$
where $F(\phi_{qN},\phi_{q\pi})$ is as defined above and 
$G(\phi_{qN},\phi_{q\pi}) = \cos{\phi_{qN}}\sin{\phi_{q\pi}}
-\sin{\phi_{qN}}\cos{\phi_{q\pi}}$.

With these two partial derivatives in hand, the derivatives of $t$ with 
respect to the pion angles are relatively simple:
$$
\frac{\partial{t}}{\partial{\cos{\theta_{q\pi}}}} = 
2p_{\pi}q + 2(q\cos{\theta_{q\pi}}-\omega\frac{p_{\pi}}{E_{\pi}})
\frac{\partial{p_{\pi}}}{\partial{\cos{\theta_{q\pi}}}}
$$
and
$$
\frac{\partial{t}}{\partial{\phi_{q\pi}}} = 
2(q\cos{\theta_{q\pi}}-\omega\frac{p_{\pi}}{E_{\pi}})
\frac{\partial{p_{\pi}}}{\partial{\phi_{q\pi}}}.
$$

The derivatives of $\phi^*_{q\pi}$ are somewhat involved.  The first step is
to describe how it is calculated.  First, all relevant vectors are described 
in the lab using what I will call the ``q'' coordinate system.  This coordinate
system is defined such that the z-axis lies along $\vec{q}$.  Then, since
$\vec{q}$ and $\vec{P}_{beam}$ describe the scattering plane, the y-axis is 
defined to be in the direction $\vec{q} \times \vec{P}_{beam}$, and then the
x-axis is $\vec{y} \times \vec{z}$.  Then $\phi_{q\pi}$ is simply 
$\arctan{\frac{p_y}{p_x}}$.  Typically, when the target is stationary, the 
boost to the center of mass is along $\vec{q}$, so that $\phi_{q\pi}$ is 
the same in the lab and the center of mass.  When the target is allowed to 
move, however, we do not in general boost along $\vec{q}$, so we must 
recalculate  $\phi^*_{q\pi}$.

To calculate $\phi^*_{q\pi}$, we define the ``q'' system as before, but in 
terms of all variables in the center of mass.  Just to make the calculation 
clear, I will spell out the steps.  Assuming I start with some $p_{\pi}^x$ and
$p_{\pi}^y$ in the ``q'' system in the lab, I must first boost to the center of
mass (note that I will drop the $\pi$ subscript - unless specifically noted, 
all momenta refer to the pion).  In this calculation, I write the explicit 
angular dependence of $p_x$ ($=p\sin{\theta}\cos{\phi}$), etc. so that it is 
clearer how to take the partial derivatives later.
$$
p_x^* = p\sin{\theta_{q\pi}}\cos{\phi_{q\pi}} + 
p\frac{\gamma-1}{\beta^2}\beta_x(\beta_x\sin{\theta_{q\pi}}\cos{\phi_{q\pi}}+
\beta_y\sin{\theta_{q\pi}}\sin{\phi_{q\pi}}+\beta_z\cos{\theta_{q\pi}}) - 
\gamma\beta_xE
$$

$$
p_y^* = p\sin{\theta_{q\pi}}\sin{\phi_{q\pi}} + 
p\frac{\gamma-1}{\beta^2}\beta_y(\beta_x\sin{\theta_{q\pi}}\cos{\phi_{q\pi}}+
\beta_y\sin{\theta_{q\pi}}\sin{\phi_{q\pi}}+\beta_z\cos{\theta_{q\pi}}) - 
\gamma\beta_yE
$$

$$
p_z^* = p\cos{\theta_{q\pi}} + 
p\frac{\gamma-1}{\beta^2}\beta_z(\beta_x\sin{\theta_{q\pi}}\cos{\phi_{q\pi}}+
\beta_y\sin{\theta_{q\pi}}\sin{\phi_{q\pi}}+\beta_z\cos{\theta_{q\pi}}) - 
\gamma\beta_zE
$$

Note that the above center of mass quantities are written in terms of the lab
``q'' coordinate system and must be rotated to the center of mass ``q'' 
coordinate system as described previously. This rotation is totally determined
by the boosted beam and q vectors (so do not depend on the pion angles).
$$
p_x^{*'} = p_x^* R_{xx} + p_y^* R_{xy} + p_z^* R_{xz}
$$
$$
p_y^{*'} = p_x^* R_{yx} + p_y^* R_{yy} + p_z^* R_{yz}
$$
Then, finally, we can calculate $\phi_{q\pi}^*$.
$$
\phi_{q\pi}^* = \arctan{\frac{p_y^{*'}}{p_x^{*'}}}
$$

Now, at last we can calculate the partial derivatives for the Jacobian.  First
I use the fact that,
$$
\frac{\partial{\tan{\phi_{q\pi}}}}{\partial{R}} = 
(1+\tan^2{\phi_{q\pi}})\frac{\partial{\phi_{q\pi}}}{\partial{R}}.
$$
This gives,
$$
\frac{\partial{\phi_{q\pi}}}{\partial{R}} = 
\frac{1}{1+\frac{p_y^{*'}}{p_x^{*'}}} 
\frac{ \partial{ (\frac{p_y^{*'}}{p_x^{*'}} }) }{\partial{R}}
$$
$$
\frac{\partial{\phi_{q\pi}}}{\partial{R}} =
\frac{p_x^{*'} \frac{\partial{p_y^{*'}}}{\partial{R}} - 
      p_y^{*'} \frac{\partial{p_x^{*'}}}{\partial{R}} }{(p_x^{*'})^2 + 
 (p_y^{*'})^2}
$$
So, the derivative of $\phi^*_{q\pi}$ with respect to some lab variable can
be done using the above formula, with the expressions for the $p_n^{*'}$ 
derived above and assuming that the $R_{nn}$ and $\beta_n$ do not depend on
the pion angles (which they should not because they are calculated completely
from initial quantities).  I won't list every derivative here, but just as an 
example, I will write out one in all its glory.
$$
\frac{ \partial{p_y^{*'}}}{\partial{\phi_{q\pi}}} = 
R_{yx}\frac{\partial{p_x^*}}{\partial{\phi_{q\pi}}} + 
R_{yy}\frac{\partial{p_y^*}}{\partial{\phi_{q\pi}}} +
R_{yz}\frac{\partial{p_z^*}}{\partial{\phi_{q\pi}}}
$$
where,
$$
\frac{\partial{p_x^*}}{\partial{\phi_{q\pi}}} = 
-p\sin{\theta_{q\pi}}\sin{\phi_{q\pi}} + 
p \frac{\gamma-1}{\beta^2} \beta_x \sin{\theta_{q\pi}} 
(\beta_y\cos{\phi_{q\pi}} - \beta_x\sin{\phi_{q\pi}}) + 
(\frac{p_x^*+\gamma\beta_xE}{p} - \gamma\beta_x\frac{p}{E})
\frac{\partial{p}}{\partial{\phi_{q\pi}}},
$$
$$
\frac{\partial{p_y^*}}{\partial{\phi_{q\pi}}} = 
p\sin{\theta_{q\pi}}\cos{\phi_{q\pi}} + 
p \frac{\gamma-1}{\beta^2} \beta_y \sin{\theta_{q\pi}} 
(\beta_y\cos{\phi_{q\pi}} - \beta_x\sin{\phi_{q\pi}}) + 
(\frac{p_y^*+\gamma\beta_yE}{p} - \gamma\beta_y\frac{p}{E})
\frac{\partial{p}}{\partial{\phi_{q\pi}}}
$$
and,
$$
\frac{\partial{p_z^*}}{\partial{\phi_{q\pi}}} = 
p \frac{\gamma-1}{\beta^2} \beta_z \sin{\theta_{q\pi}} 
(\beta_y\cos{\phi_{q\pi}} - \beta_x\sin{\phi_{q\pi}}) + 
(\frac{p_z^*+\gamma\beta_xE}{p} - \gamma\beta_z\frac{p}{E})
\frac{\partial{p}}{\partial{\phi_{q\pi}}}
$$

\begin{thebibliography}{99}
\bibitem{Brauel} P. Brauel {\it et al.},
``Electroproduction Of Pi+ N, Pi- P And K+ Lambda, K+ Sigma0 Final States
                  Above The Resonance Region,"
Zeit. Phys. {\bf C3}, 101 (1979).

\bibitem{Tiator}
D.~Drechsel, O.~Hanstein, S.S.~Kamalov and L.~Tiator,
``A Unitary isobar model for pion photoproduction and electroproduction on
                  the proton up to 1-GeV,"
Nucl. Phys. {\bf A645}, 145 (1999)
nucl-th/9807001.
\end{thebibliography}

\end{document}
