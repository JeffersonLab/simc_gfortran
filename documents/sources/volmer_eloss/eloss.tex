\tolerance=10000
\documentstyle[12pt,epsf]{article}
\parindent = 12pt
\voffset = -1in
\textheight = 8.5in
%\pagestyle{empty}

\begin{document}

\title{Calculation of Energy Losses in Hall C's Replay Engine }
\author{Jochen Volmer}
%       {\it Faculteit Natuurkunde en Sterrenkunde}\\
%       {\it Vrije Universiteit}\\
%       {\it De Boelelaan 1081, 1081 HV Amsterdam, Netherland}}
\date{\today}
\maketitle

\section{Introduction}
\hspace*{\parindent}

One of the tasks of Hall C's replay engine is to reconstruct all kinematics
variables for an event, singles or coincidence, at the point of interaction
(vertex). The calculation of the energy loss suffered by electrons and heavier 
ionizing particles in the passage through matter has to be accounted for in 
this reconstruction. Depending on where it is called from, the subroutine 
{\tt total\_eloss.f} traces either the incoming (beam) electron through the 
target entrance window and the target liquid (in case of a cryotarget) or 
the outgoing electron and hadron through the various gases and windows into 
the spectrometers. In the case of the incoming beam, {\tt total\_eloss} is
called once upon runstart, and the calculated beam energy loss is subtracted 
from the original beam energy. In the case of the particles entering the
spectrometers, {\tt total\_eloss} is called on an event by event basis, and
the calculated energy loss is {\em added} to the reconstructed particle
energy, in order to derive the energy of the particle at the vertex.

\section{Calculation of the Energy Loss}
%\hspace*{\parindent}

\subsection{Hadrons}
\hspace*{\parindent}

In the engine, the calculation of the energy loss for hadrons is not specified for a
specific type of hadrons. The routine {\tt loss} is called with the particle velocity 
$\beta_p$, which is calculated in {\tt h\_physics} or {\tt s\_physics} from the momentum 
and mass of the particle. Here, of course, an assumption has been made for the nature of 
the particle, and the energy loss calculation is correct only for the particle of 
interest.\\
The formula used in {\tt loss} is \cite{engine}:

\begin{equation}
\label{eq:hadronelossincode}
{E_{lost}} = -t\cdot\frac{dE}{dx} = -t \cdot 2 \cdot K\cdot \frac{Z}{A} \frac{1}{\beta^2}
\cdot \left( \ln{ \frac{2 m_e c^2 \beta^2 \gamma^2}{I}} - \beta^2 \right).
\end{equation}

Here, $K$ is 0.1536 GeVcm$^2$/g, $Z$ and $A$ are the atomic number and mass of
the medium, $t$ the thickness of the medium in (g/cm$^2$), $\beta$ the velocity $(v/c)$
of the particle, $\gamma=1/\sqrt{1-\beta^2}$, $m_e c^2$ the electron mass in MeV. $I$ 
is the mean excitation energy: $I$=21.8 eV for hydrogen, $I=16\cdot z^{0.9}$ for heavier 
target media. It is not known where this parametrization of $I$ came from. A better
parametrization of $I$ is given in Leo \cite{leo}: $I=12Z+7$ eV for $1<Z<13$, 
$I=9.76Z+58.8Z^{-0.19}$ eV for $Z\geq13$.\\

Eq.~\ref{eq:hadronelossincode} has to be compared with the Bethe-Bloch formula as 
presented by the Particle Data Group \cite{PDG54}:

\begin{equation}
\label{eq:BBPDG54221}
-\frac{dE}{dx}=K z^2 \frac{Z}{A} \frac{1}{\beta^2} \cdot \left[ 
  \frac{1}{2} \ln{\frac{2 m_e c^2 \beta^2 \gamma^2 T_{max}}{I^2}} 
  - \beta^2 - \frac{\delta}{2} \right].
\end{equation}

Here, most of the variables are defined as above. $ze$ is the charge of the incoming
particle, $m_e c^2$ is 0.511 MeV, and $T_{max}=\frac{2 m_e c^2 \beta^2 \gamma^2}
{1+2\gamma m_e/M+(m_e/M)^2}$ is the maximum energy transfer from a heavy particle to an 
electron, with $M$ the particle mass. 

\subsubsection{Corrections to Bethe-Bloch}
\label{subsubsection:hadroncorrections}
\hspace*{\parindent}

The PDG and Leo \cite{leo,PDG54} mention two corrections to the basic Bethe-Bloch formula.
One is a shell-correction, which treats effects at very low particle momenta, when the
particle velocity is comparable or lower than the orbital velocity of the bound atomic 
electrons. The particle momenta where this correction takes effect are so low that it can
be safely omitted in the context of CEBAF energies.\\
The other correction is the so-called density correction $\delta$ 
(Eq.~\ref{eq:BBPDG54221}). As the particle energy increases, its electric field flattens
and extends, so that the distant-collision contribution to the energy loss increases as
$\delta/2=\ln{(\hbar \omega_p / I)}+\ln{\beta \gamma}-1/2$. $\beta\gamma = p/Mc$ is the 
particle momentum in terms of its mass, $\hbar \omega_p$ is the so-called plasma energy, 
parametrized as $28.816 \cdot \sqrt{\rho <Z/A>}$ eV \cite{leo,PDG54}. The term with 
$\ln{(\hbar\omega_p/I)}$ accounts for the polarizability of the medium. However, this 
parametrization of the density correction is only valid at large $\beta\gamma$.\\ 
For electrons, this parametrization is valid almost always (see section 
\ref{subsection:electrons}). For hadrons, however, a parametrization is necessary.
The Particle Data Group \cite{PDG54} cites Sternheimer's parametrization for nonconductors
\cite{Sternheimer}:
\begin{equation}
\label{eq:denscorr}
 \delta= \left\{ 
              \begin{array}{ll}
	        2(\ln{10})x-\bar{C} & if\hspace{6pt} x \geq x_1; \\
		2(\ln{10})x-\bar{C}+a(x_1-x)^k & if\hspace{6pt} x_0 \leq x < x_1; \\
		0 & if\hspace{6pt} x<x_0
	      \end{array}
           \right. 
\end{equation}
Here, $x=\log_{10}{\beta\gamma}$, $\bar{C}=-(2\ln{(I/(\hbar\omega_p))}+1)$. $\bar{C}$,
$x_0$, $x_1$, $a$ and $k$ are material dependent constants. Leo \cite{leo} cites the
values of these constants for a number of materials. $x_0$ is usually around zero, with
the exception of $N_2$, $O_2$ and air, where it is around 1.7. $x_1$ is mostly 
between 2.5 and 3.5, again with the exception of $N_2$, $O_2$ and air, where it is around
4.2. Values of $k$ range between 2.9 and 3.6. The value of $a$ is chosen such that it
provides a smooth passage from $x<x_0$ to $x>x_1$.\\
In order to parametrize this in a simple way for the replay code, a general
parametrization for all materials was chosen: $x_0=0$, $x_1=3$, $k=3$ and $a=-C_0/27$.\\

\subsubsection{Comparison of the Energy Loss Calculations for Hadrons}
\hspace*{\parindent}

\begin{figure}
\hspace{0.0in}
\epsfxsize=5.0in
\epsfysize=6.0in
\epsffile{hadroneloss.ps}
\caption{Energy loss of hadrons in LH$_2$ (upper left), graphite (upper right), aluminum 
(lower left) and copper (lower right). The energy loss as calculated by the standard
engine (dashed-dotted line) is compared with the standard Bethe-Bloch formula, with two
parametrizations for the density correction $\delta$ (solid and dashed lines). See text 
for explanations on the density correction. The momentum scale is $\beta \gamma=p/Mc$, 
which is the particle momentum in terms of the particle mass.}
\label{fig:hadronloss}
\end{figure}

\begin{figure}
\hspace{0.0in}
\epsfxsize=5.0in
\epsfysize=6.0in
\epsffile{hadronelossrel.ps}
\caption{The difference in the calculation of the energy loss of hadrons passing through
matter in \%. The figure on top shows the overprediction of the energy loss at low 
particle momenta when the density correction is done in a high energy limit expression. 
The bottom figure shows the error made by the standard replay engine formula, which does
not contain density corrections. BB stands for Bethe-Bloch with Sternheimer's 
parametrization of the density correction.}
\label{fig:hadronlossrel}
\end{figure}

The different results of Equations \ref{eq:hadronelossincode} and \ref{eq:BBPDG54221} 
with different parametrizations of the density correction are shown in 
Fig.~\ref{fig:hadronloss} for four different materials. Most significant are the effect 
of the density correction at high energies. Fig.~\ref{fig:hadronlossrel} shows what errors
are made by either omitting the density correction altogether (bottom) or by putting in
the high energy density correction at low energies (i.e. without using Sternheimer's 
parametrization). The result of this exercise is that the energy loss calculation in
the code in its old state (7/98) does a reasonable job ($\pm$10\%) for protons up to 
about 10 GeV/c and for pions up to 1.5 GeV/c. In order to do a better job for pions, and
in the light of the upcoming upgrade of the CEBAF accelerator, the density correction in
the Sternheimer parametrization clearly should be coded into the engine.\\
The error made by making a general parametrization for the $x_0<x<x_1$ part will be at
least an order of magnitude lower than the density correction itself and can therefore
be tolerated.

\newpage

\subsection{Electrons} 
\label{subsection:electrons}
%\hspace*{\parindent}

\subsubsection{Corrections to Bethe-Bloch for Electrons}
\hspace*{\parindent}

For electrons, the matter is more complicated than for hadrons \cite{leo}:

\begin{quote}
``While the basic mechanism of collision loss [..] for heavy charged particles is
also valid for electrons, the Bethe-Bloch formula must be modified somewhat for
two reasons. One, [..], is their small mass. The assumption that the incident
particle remains undeflected during the collision process is therefore invalid.
The second is that for electrons the collisions are between identical particles,
so that the calculation must take into account their indistinguishability. These
considerations change a number of terms in the formula, in particular, the maximum
allowable energy transfer becomes $W_{max}=T_e/2$ where $T_e$ is the kinetic energy
of the incident electron or positron. [..]''
\end{quote}

The Bethe-Bloch equation (Eq.~\ref{eq:BBPDG54221}) calculates a mean energy loss.
A relativistic correction for electrons to the Bethe-Bloch equation has been devised
by Bethe \cite{leo,musiol}. This not only takes into account the changed $T_{max}$, but
also the effect that a collision between electrons can result in a big change in the
direction of the particle momentum. With, among others, this correction, the formula 
reads

\begin{eqnarray}
\label{eq:bethemusiol}
-\frac{dE}{dx} = & K \frac{Z}{A} \frac{1}{\beta^2} \cdot 
  \left[
	\ln{\frac{m_e c^2 \beta^2 \gamma^2 E_e}{2 I^2}} + (1-\beta^2) 
 	- \frac{2\gamma-1}{\gamma^2} \ln{2} + \frac{1}{8}\left( 
 		\frac{\gamma-1}{\gamma}
	\right)^2 - \delta
  \right] \\
\label{eq:bethe}
 = & K \frac{Z}{A} \frac{1}{\beta^2} \cdot 
  \left[
   	\ln{\frac{\tau^2(\tau+2)}{2(I/m_e c^2)^2}}
        +(1-\beta^2)+\frac{\frac{\tau^2}{8}-(2\tau+1)\ln{2}}{(\tau+1)^2} 
	- \delta
  \right],
\end{eqnarray}
 
with K is defined as in Eq.~\ref{eq:BBPDG54221} and $E_e$ the relativistic kinetic
energy of the electron, $m_e c^2 (\gamma-1)$, and $\tau=\gamma-1$ is the kinetic energy
of the electron in terms of the electron mass. $\delta$ is the density correction 
discussed in section \ref{subsubsection:hadroncorrections}.
In the case of electrons at CEBAF energies (between 0.1 and 4 GeV/c), $x=\log_{10}{\beta
\gamma}$ is big enough (2.3-3.9) that the density correction can be applied without 
Sternheimer's parametrization. The error made here is 0.1-0.2\% in the worst case.

\newpage
\subsubsection{Most Probable versus Average Energy Loss}
\hspace*{\parindent}

So far, the engine ({\tt total\_eloss}) had used a most probable energy loss 
\cite{engine}, as measured by O'Brien {\em et al.} for 50 and 100 MeV electrons on a 
number of solid targets with thicknesses between 48 and 614 mg/cm$^2$ \cite{obrien}:

\begin{equation}
\label{eq:obrien}
E_{p}=A^\prime t \left(K+ \ln {\frac{t}{\rho}} \right).
\end{equation} 

Here, $A^\prime= 0.1536 Z/A$ MeV cm$^2$/g., $K=19.26$, $t$ is the target thickness
in (g/cm$^2$) and $\rho$ is the target density in (g/cm$^3$). This equation is based
on ``Landau straggling''. In a Landau straggling distribution, the average energy
loss is always greater than the most probable energy loss. This equation also does not 
take into account the energy dependence of the energy loss due to relativistic effects in 
the medium.\\
A problem with using the most probable energy loss is that the Landau distribution in the 
experiment is convoluted with resolutions of gaussian shape which are much broader than
the Landau distribution. The convolution of a narrow Landau distribution and
a wide gaussian distribution will still have the shape of a gaussian distribution, and
it will be shifted by an amount closer to the average energy loss than the most probable
energy loss\footnote{Thanks to Dave Gaskell for actually proving that!} \cite{Gaskell}. 

\subsubsection{Comparison of the eloss calculations for electrons}
\hspace*{\parindent}

Fig.~\ref{fig:electronloss} shows the difference of the numerical outcome of energy
loss calculations after Bethe-Bloch, Bethe and O'Brien for a number of targets.
Both the Bethe-Bloch calculation and the Bethe calculation with the relativistic 
correction include the density correction. The dashed-dotted line represents the O'Brien
formula for the most probable energy loss (Eq.~\ref{eq:obrien}). The dashed line 
represents the usual Bethe-Bloch calculation, with $T_{max}=T/2$ 
(Eq.~\ref{eq:BBPDG54221}). The solid curve represents Bethe's parametrization of the 
relativistic rise for electrons (Eq.~\ref{eq:bethe}). The latter two calculate the 
average energy loss.\\
Fig.~\ref{fig:electronrelloss} shows the ratios of the calculated energy losses for a 
number of targets. It becomes clear that at the momentum where the electron
is minimum ionizing the difference between the Bethe-Bloch-like calculations of the
{\em average} energy loss and O'Briens formula for the {\em most probable} energy loss
is roughly 20-35\%. However, in the region of interest, between 100 MeV/c and 4 GeV/c,
the difference is significant: the ratio is between 1.3 and 1.8, depending on the 
material, the thickness and the electron momentum. The difference is most significant for
heavy materials and thin layers.\\
The relativistic rise of the average energy loss in the region of interest is roughly 
15-20\%. We do not know of data
for a momentum-dependent shift of the most probable energy loss. One can speculate that 
the behaviour of the average and of the most probable energy losses are similar. 
Bethe's parametrization of the relativistic rise for electrons (Eq.~\ref{eq:bethe})
contributes a 4-6\% effect on top of the $T_{max}$ corrected Bethe-Bloch formula 
(Eq.~\ref{eq:BBPDG54221}).  

\section{Conclusion and Actions Taken}
%\hspace*{\parindent}

The treatment of energy loss due to the passage of ionizing particles through matter
as done in the engine has been outlined as well as corrections to it. In the case of 
hadrons, the difference between a full Bethe-Bloch (Eq.~\ref{eq:BBPDG54221}) with
Sternheimer's parametrization for the density correction \cite{Sternheimer}
and the simpler formula that is being used at the moment (Eq.~\ref{eq:hadronelossincode})
is almost insignificant at low momenta. At high momenta, the density correction will
contribute up to 15\% to the energy loss, especially for pions, and will therefore be 
taken into account in the engine code in the future.\\
In the case of electrons, there are two corrections to the Bethe-Bloch formula
competing with the O'Brien formula, both of which predict a relativistic rise of the
energy loss. Unfortunately, one involuntarily compares apples and oranges here, since
the Bethe-Bloch type formulas treat average energy losses, while O'Briens formula the
most probable one.\\
Our expected resolution is bigger than the Landau distribution.
It can be argued that the convolution of a shifted narrow Landau distribution with a broad
gaussian resolution will still look like a gaussian, shifted by the average of the Landau
distribution, not its maximum value. This and the lack of a relativistic rise of the
O'Brien formula leads us to replace O'Briens formula (Eq.~\ref{eq:obrien}) with Bethe's 
formula (Eq.~\ref{eq:bethe}) in the engine code.

\section{Acknowledgements}
%\hspace*{\parindent}

I owe thanks to Rolf Ent and Dave Mack for fruitful discussions, and especially to Dave 
Gaskell, who actually verified my calculations and caught my mistakes.

\begin{thebibliography}{99}
\bibitem{engine} Hall C replay engine, source code
\bibitem{leo} W.R.~Leo, Techniques for Nuclear and Particle Physics Experiments, Second
Edition, Springer Verlag Berlin, Heidelberg 1987, 1994
\bibitem{PDG54} Particle Data Group, Eur. Phys. Jour. C3(1998), section 23. 
\bibitem{Sternheimer} R.M.~Sternheimer, Phys. Rev. 88(1952)851
\bibitem{musiol} G.~Musiol, J.~Ranft, R.~Reif, D.~Seeliger, Nuclear- and 
Elementary Particle Physics, VCH Publishers (UK), 1988
\bibitem{obrien} James T.~O'Brien {\em et al.}, Phys. Rev. C9(1974)1418
\bibitem{Gaskell} Dave Gaskell, priv.\ comm.
\end{thebibliography}

\begin{figure}
\hspace{0.0in}
\epsfxsize=5.0in
\epsfysize=6.0in
\epsffile{elosselectron.ps}
\caption{-dE/dx for electrons in LH$_2$, graphite and copper. Plotted are the 
thickness-independent Bethe-Bloch curve with relativistic corrections (solid line), the 
Bethe calculation with only $T_{max}=T/2$ (long dashed) and the O'Brien formula 
(dash-dotted).}
\label{fig:electronloss}
\end{figure}

\begin{figure}
\hspace{0.0in}
\epsfxsize=5.0in
\epsfysize=6.0in
\epsffile{elosselrel.ps}
\caption{Ratio of the calculations of electron energy losses in LH$_2$, graphite and
copper. Top left: ratio for Bethe-Bloch formula and O'Briens formula; top right: ratio for
Bethe with relativistic corrections and O'Brien; bottom: ratio between Bethe with
relativistic corrections and Bethe-Bloch. The Bethe-Bloch calculation is the original
one with only $T_{max}$ changed to $T/2$.}
\label{fig:electronrelloss}
\end{figure}

\newpage

\appendix

\section{Energy Loss in the Code}
\label{section:elossinthecode}
%\hspace*{\parindent}

\subsection{The Old Version of {\tt loss}}
\hspace*{\parindent}

{\tt
      subroutine loss(electron,z,a,thick,dens,beta,e\_loss)\\
**-------------------------------------------------------------\\
**-         Prototype C function \\
**- \\
**-\\
**-    Purpose and Method :  Calculate energy loss \\
**-    \\
**-    Output: -\\
**-    Created   1-Dec-1995  Rolf Ent\\
**-   \\
**-    Verification:  The non-electron portion on this subr. is \\
**-		     Bethe-Bloch equation (Physial Review D vol.50 \\
**-		     (1994) 1251 with theassumpution that the \\
**-		     denominator in Tmax is equal to 1.0. By not\\
**-	  	     making this assumption the change in the dE/dx \\
**-		     is by a factor on the order 10e-3 less for a \\
**-		     pi+ particle.\\
**-                   K. Vansyoc 3/4/98 16:50\\
**-----------------------------------------------------------------*\\
      IMPLICIT NONE\\
      SAVE\\
**\\
      include 'gen\_data\_structures.cmn'\\
**\\
      LOGICAL electron\\
      REAL*4 eloss,z,a,thick,dens,beta,e\_loss\\
      REAL*4 icon\_ev,me\_ev\\
      REAL*4 particle\\
      parameter (me\_ev = 510999.)\\
**\\
 \\
 91   format(7(A10))\\
 90   format(7(2x,f8.5))\\
      e\_loss = 0.0\\
      eloss  = 0.0\\
 \\
*******************************************************\\
** for debugging print out all variables that have been \\
** passed on to loss\\
*******************************************************\\
 \\
** electron\\
      if (electron) then\\
        if(thick.gt.0.0.and.dens.gt.0.0)then\\
            eloss = 0.1536e-03*z/a*thick*(19.26 + log(thick/dens))\\
         endif\\
      endif\\
** proton\\
      if(.not.electron) then\\
         icon\_ev = 16.*z**0.9\\
         if (z.lt.1.5) icon\_ev = 21.8\\
         if(thick.gt.0.0.and.beta.gt.0.0.and.beta.lt.1.0)then\\
          eloss = log(2.*me\_ev*beta*beta/icon\_ev/(1.-beta*beta))\\
     \&                - beta*beta\\
          eloss = 2.*0.1536e-03*z/a*thick/beta/beta * eloss\\
         endif\\
      endif\\
 \\
** units should be in GeV\\
      e\_loss = eloss\\
 \\
      if (gelossdebug.ne.0) then\\
         particle=0.0\\
         if (electron) particle=1.0\\
         write(6,91) 'electron?','ztgt','atgt','thick','dens','beta','e\_loss'\\
         write(6,90) particle,z,a,thick,dens,beta,e\_loss\\
      endif\\
 \\
      RETURN\\
      END\\
 \\
 \\
}

\newpage

\subsection{The New Version of {\tt loss}}
\hspace*{\parindent}
%The new version of the subroutine {\tt loss}: \hspace{20pt}

{\tt
      subroutine loss(electron,z,a,thick,dens,velocity,e\_loss)\\
**-------------------------------------------------------------\\
**-         Prototype C function \\
**- \\
**-\\
**-    Purpose and Method :  Calculate energy loss \\
**-    \\
**-    Output: -\\
**-    Created   1-Dec-1995  Rolf Ent\\
**-   \\
**-    Verification:  The non-electron portion on this subr. is \\
**-		     Bethe-Bloch equation (Physial Review D vol.\\
**-		     50 (1994) 1251 with full calculation of \\
**-		     Tmax and the density correction. The electron \\
**-		     part has been switched from O'Brien, Phys. \\
**-		     Rev. C9(1974)1418, to Bethe-Bloch with \\
**-		     relativistic corrections and density density \\
**-		     correction, Leo, Techniques for Nuclear and\\
**-		     Particle Physics Experiments.\\
**-                   J. Volmer 8/2/98 16:50\\
**-----------------------------------------------------------------*\\
      IMPLICIT NONE\\
      SAVE\\
**\\
      include 'gen\_data\_structures.cmn'\\
      include 'hms\_data\_structures.cmn'\\
      include 'sos\_data\_structures.cmn'\\
**\\
      LOGICAL electron\\
      REAL*4 eloss,z,a,thick,dens,beta,e\_loss\\
      REAL*4 icon\_ev,me\_ev\\
      REAL*4 icon\_gev,me\_gev\\
      REAL*4 particle\\
      REAL*4 denscorr,hnup,c0,log10bg,pmass,tmax,gamma,velocity\\
      REAL*4 tau,betagamma\\
      parameter (me\_ev = 510999.)\\
      parameter (me\_gev = 0.000510999)\\
**\\
 \\
 91   format(7(A10))\\
 90   format(7(2x,f8.5))\\
      e\_loss = 0.0\\
      eloss  = 0.0\\
 \\
*******************************************************\\
** for debugging print out all variables that have been\\
** passed on to loss\\
*******************************************************\\
 \\
*******************************************************\\
** calculate the mean excitation potential I in a newer \\
** parametrization given in W.R. Leo's Techniques for \\
** Nuclear and Particle Physics Experiments\\
*******************************************************\\
 \\
      if (z.lt.1.5) then\\
         icon\_ev = 21.8\\
      elseif (z.lt.13) then\\
         icon\_ev = 12.*z+7.\\
      elseif (z.ge.13) then\\
         icon\_ev = z*(9.76+58.8*z**(-1.19))\\
      endif\\
      icon\_gev = icon\_ev*1.0e-9\\
 \\
***********************************************\\
** extract the velocity of the particle:\\
**     hadrons:   velocity = beta\\
**     electrons: velocity = log\_10(beta*gamma)\\
***********************************************\\
 \\
      if (electron) then\\
         log10bg=velocity\\
         betagamma=exp(velocity*log(10.))\\
         beta=betagamma/(sqrt(1+betagamma**2))\\
         gamma=sqrt(1.+betagamma**2)\\
         tau=gamma-1.\\
      elseif (.not.electron) then\\
         beta=velocity\\
         gamma=1./sqrt(1.-beta**2)\\
         betagamma=beta*gamma\\
         log10bg=log(betagamma)/log(10.)\\
         tau=gamma-1.\\
      endif\\
 \\
*******************************************************\\
** calculate the density correction, as given in Leo,\\
** with Sternheimer's parametrization\\
** I is the mean excitation potential of the material\\
** hnup= h*nu\_p is the plasma frequency of the material\\
*******************************************************\\
 \\
      denscorr=0.\\
      HNUP=28.816E-9*sqrt(DENS*Z/A)\\
 \\
      C0=-2*(log(icon\_gev/hnup)+.5)\\
 \\
      if(log10bg.lt.0.) then\\
         denscorr=0.\\
      elseif(log10bg.lt.3.) then\\
         denscorr=C0+2*log(10.)*log10bg+abs(C0/27.)*(3.-log10bg)**3\\
      else\\
         denscorr=C0+2*log(10.)*log10bg\\
      endif\\
 \\
****************************************************************\\
** for hadrons: calculate the maximum possible energy transfer \\
**		to an orbital electron, find out what the hadron \\
**		mass is\\
****************************************************************\\
 \\
      pmass=0.\\
      if (.not.electron) then\\
         pmass=max(hpartmass,spartmass)\\
         if (pmass.lt.2*me\_gev) pmass=0.5\\
         tmax=2*me\_gev*beta**2*gamma**2/\\
     >        (1+2*gamma*me\_gev/pmass+(me\_gev/pmass)**2)\\
      endif\\
 \\
**********************************************\\       
** now calculate the energy loss for electrons \\
**********************************************\\
** electron\\
      if (electron) then\\
         if(thick.gt.0.0.and.dens.gt.0.0)then\\
            eloss=0.1535e-03*z/a*thick/beta**2*(\\
     >           log(tau**2*(tau+2.)/2./(icon\_gev/me\_gev)**2)\\
     >           +1-beta**2+(tau**2/8-(2*tau+1)*log(2.))/(tau+1)**2\\
     >           -(-(2*log(icon\_gev/hnup)+1)+2*log(betagamma)))\\
         endif\\
 \\
*jv        if(thick.gt.0.0.and.dens.gt.0.0)then\\
*jv           eloss = 0.1536e-03*z/a*thick*(19.26 + log(thick/dens))\\
*jv         endif\\
 \\
      endif\\
 \\
********************************************\\      
** now calculate the energy loss for hadrons \\
********************************************\\
** proton\\
      if(.not.electron) then\\
 \\
**jv         icon\_ev = 16.*z**0.9\\
**jv         if (z.lt.1.5) icon\_ev = 21.8\\
 \\
         if(thick.gt.0.0.and.beta.gt.0.0.and.beta.lt.1.0)then\\
 \\
            eloss = 2.*0.1535e-3*Z/A*thick/beta**2*(\\
     >           .5*log(2*me\_gev*beta**2*gamma**2*tmax/icon\_gev**2)\\
     >           -beta**2-denscorr/2.)\\
 \\
**jv          eloss = log(2.*me\_ev*beta*beta/icon\_ev/(1.-beta*beta))\\
**jv     \&                - beta*beta\\
**jv          eloss = 2.*0.1536e-03*z/a*thick/beta/beta * eloss\\
 \\
         endif\\
      endif\\
 \\
** units should be in GeV\\
      e\_loss = eloss\\
 \\
      if (gelossdebug.ne.0) then\\
         particle=0.0\\
         if (electron) particle=1.0\\
         write(6,91) 'electron?','ztgt','atgt','thick','dens','velocity'\\
     > 		,'e\_loss'\\
         write(6,90) particle,z,a,thick,dens,velocity,e\_loss\\
         write(6,'(4A10)') 'velocity','beta','pmass','denscorr'\\
         write(6,'(6(2x,f8.5))') velocity,beta,pmass,denscorr\\
         write(6,'(6A10)') 'betagamma','log10bg','tau','gamma',\\
     >		'icon\_ev','hnup (eV)'\\
         write(6,'(6(2x,F8.3))') betagamma,log10bg,tau,gamma,icon\_ev,hnup*1e9\\
      endif\\
 \\
      RETURN\\
      END\\
 \\
 \\
}

%\section{Average Energy Loss versus Most Probable Energy Loss}
%\label{section:appendixaveragevsmostprobable}

%\begin{figure}
%\hspace{0.0in}
%\epsfxsize=5.0in
%\epsfysize=6.0in
%\epsffile{eloss.nb.ps}
%\caption{Ratio of the calculations of electron energy losses in LH$_2$, graphite and
%copper. Top left: ratio for Bethe-Bloch formula and O'Briens formula; top right: ratio for
%Bethe with relativistic corrections and O'Brien; bottom: ratio between Bethe with
%relativistic corrections and Bethe-Bloch. The Bethe-Bloch calculation is the original
%one with only $T_{max}$ changed to $T/2$.}
%\label{fig:elossnb}
%\end{figure}


\end{document}
